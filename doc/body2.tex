% !TeX root = ../main.tex

%%%%%%%%%%%%%%%%%%%%%%%%%%%%%%%%%%%%%%%%%%%%%%%%%%%%%%%%%%%%%%%%%%%%%%%%%%%%%%%%%%
%%																				%%
%% File name: 		body.tex													%%
%% Project name:	Applications in Deep Learning								%%
%% Type of work:	Advanced Seminar											%%
%% Author:			Hannes Bohnengel											%%
%% Mentor:			Debayan Roy													%%
%% Date:			27 May 2017													%%
%% University:		Technical University of Munich								%%
%% Comments:		Created in texstudio with tab width = 4						%%
%%																				%%
%%%%%%%%%%%%%%%%%%%%%%%%%%%%%%%%%%%%%%%%%%%%%%%%%%%%%%%%%%%%%%%%%%%%%%%%%%%%%%%%%%

\section{Speech Synthesis}
\label{sec:speech}

\subsection{Motivation}
\label{subsec:motspeech}

Why speech synthesis is useful/important?\\What are use cases in daily life?\\Why there is need to further improve this technology?

\subsection{Conventional Approaches}
\label{subsec:convenspeech}

Brief overview of conventional approaches how to implement speech synthesis, with highlighting advantages and drawbacks.

\begin{itemize}[leftmargin=10pt]
	\item concatenative \& unit-selection
	\item formant-based
	\item diphone-based
	\item \ac{SPSS}
	\item etc. ??
\end{itemize}

\subsection{HMM based Speech Synthesis: \ac{SPSS}}
\label{subsec:hmmspeech}

Description of one approach (\ac{SPSS}) more in detail \cite{zen:statistical}.

\subsection{\ac{SPSS} with Deep Learning Models}
\label{subsec:deepspeech}

Description of the improvements of the approach in previous subsection by using deep learning models.

\subsubsection{General ways for improvement}

The effect of neural networks in statistical parametric speech synthesis \cite{hashimoto:effect}

\subsubsection{One specific approach for improvement}

Statistical parametric speech synthesis using deep neural networks \cite{zen:deepstatistical}

\section{Speech Synthesis on Embedded Devices}
\label{sec:embeddedspeech}

\subsection{Motivation}
\label{subsec:motembedded}

Why is it important to implement speech synthesis on embedded platform?\\
What needs to be thought about when dealing with embedded or mobile devices?

\subsection{\ac{HMM}-based Approach}
\label{subsec:hmmembedded}

An example of how speech synthesis can be implemented on embedded platform without deep learning (core paper~3).

\subsection{Deep Learning-based Approach}
\label{subsec:deepembedded}

An example of how speech synthesis can be implemented on embedded platform WITH deep learning (core paper 4).

\clearpage