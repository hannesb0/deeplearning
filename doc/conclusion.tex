% !TeX root = ../main.tex

%%%%%%%%%%%%%%%%%%%%%%%%%%%%%%%%%%%%%%%%%%%%%%%%%%%%%%%%%%%%%%%%%%%%%%%%%%%%%%%%%%
%%																				%%
%% File name: 		conclusion.tex												%%
%% Project name:	Applications in Deep Learning								%%
%% Type of work:	Advanced Seminar											%%
%% Author:			Hannes Bohnengel											%%
%% Mentor:			Debayan Roy													%%
%% Date:			13 July 2017												%%
%% University:		Technical University of Munich								%%
%% Comments:		Created in texstudio with tab width = 4						%%
%%																				%%
%%%%%%%%%%%%%%%%%%%%%%%%%%%%%%%%%%%%%%%%%%%%%%%%%%%%%%%%%%%%%%%%%%%%%%%%%%%%%%%%%%

\section{Conclusions}
\label{sec:conclusion}

With this paper, I provided a systematic review of the impact of deep learning models on speech synthesis on mobile devices. In this context, first, different approaches to improve the conventional \ac{HMM}-based synthesis have been pointed out, including the deployment of \acp{DNN} as acoustic models or in the parameter generation task. These approaches have lead to a significant improvement of the prediction performance and the speech quality~\cite{zen:deepstatistical, hashimoto:effect}. 
Next, I~have studied the implementation of speech synthesis algorithms on resource-constrained environments,
%In the following, the goal of modifying speech synthesis for the use in resource-constrained environments, 
such as mobile devices, 
and accordingly outlined
%is outlined by reference to 
two strategies: one with and one without the use of deep learning models. In the first one, the focus is set on adjusting different parameters and applying several optimization steps \cite{toth:optimizing} in order to achieve real-time playback. As a result, the computation time can be decreased by 65\,\% without a negligible loss of speech quality. The second approach \cite{boros:robust} tackles the dependency of network access while using speech synthesis applications on mobile devices. Therefore, the authors have suggested the use of \acp{DNN} in several parts of the front-end of a \ac{TTS} system so as to reduce the footprint size which enables the offline use of speech synthesis applications.

It is expected that the acceptance of \acp{VPA} like Apple's Siri or Amazon's Alexa will be widespread in the near future \cite{gartner:assistants}. Hence, the development of robust and resource-efficient speech synthesis methods is an essential part to meet the challenges of mobile environments. Innovative technologies like personalized speech-to-speech translation or voice cloning are only two examples of emerging techniques, wherefore speech synthesis has to be as evolved as possible~\cite{edinburgh:speech}. I~believe that the use of deep learning methods has the potential to accelerate future developments in the field of speech synthesis on mobile devices. Consequently, it is worthwhile to further improve these methods both in terms of voice quality and efficient embedded implementation. % as part of a promising research direction in the future. % Thus, this topic is a promising research direction for the future.