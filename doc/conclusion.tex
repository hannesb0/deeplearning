% !TeX root = ../main.tex

%%%%%%%%%%%%%%%%%%%%%%%%%%%%%%%%%%%%%%%%%%%%%%%%%%%%%%%%%%%%%%%%%%%%%%%%%%%%%%%%%%
%%																				%%
%% File name: 		conclusion.tex												%%
%% Project name:	Applications in Deep Learning								%%
%% Type of work:	Advanced Seminar											%%
%% Author:			Hannes Bohnengel											%%
%% Mentor:			Debayan Roy													%%
%% Date:			12 June 2017												%%
%% University:		Technical University of Munich								%%
%% Comments:		Created in texstudio with tab width = 4						%%
%%																				%%
%%%%%%%%%%%%%%%%%%%%%%%%%%%%%%%%%%%%%%%%%%%%%%%%%%%%%%%%%%%%%%%%%%%%%%%%%%%%%%%%%%

\section{Conclusions}
\label{sec:conclusion}

In this paper a systematic review of the impact of deep learning models on speech synthesis on mobile devices, is given. Therefore different approaches to improve the conventional \ac{HMM}-based synthesis are pointed out, including the deployment of \acp{DNN} as acoustic models or in the parameter generation task. Here a significant improvement of the prediction performance and the speech quality could be achieved \cite{zen:deepstatistical, hashimoto:effect}. In the following, the goal of modifying speech synthesis for the use in resource-constrained environments, such as mobile devices, is outlined by reference to two strategies, one with and one without the use of deep learning models. In the first one, the focus is set on adjusting different parameters and applying several optimization steps \cite{toth:optimizing} in order to achieve real-time playback. As a result the computation time could be decreased by 65\,\%, without a negligible loss of speech quality. The last approach \cite{boros:robust} tackled the dependency of network access, while using speech synthesis applications on mobile devices. Therefore the authors suggested the use of \acp{DNN} in several parts of the front-end of a \ac{TTS} system so as to reduce the footprint size, which enables the offline use of speech synthesis applications.

It is expected that the acceptance of \acp{VPA} like Apple's Siri or Amazon's Alexa will increase in the near future \cite{gartner:assistants}. Therefore the development of robust and resource-efficient speech synthesis methods is an essential part to meet the challenges of mobile environments. Innovative technologies like personalized speech-to-speech translation or voice cloning are only two examples of emerging techniques, wherefore speech synthesis has to be as evolved as possible~\cite{edinburgh:speech}.

\iffalse
\vspace{2em}
See \cite{edinburgh:speech}
\begin{itemize}[leftmargin=10pt]
	\item Voice cloning
	\item Voice reconstruction
	\item Personalised speech-to-speech translation
	\item Articulatory-controllable speech synthesis
\end{itemize}
\fi