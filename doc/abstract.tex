% !TeX root = ../main.tex

%%%%%%%%%%%%%%%%%%%%%%%%%%%%%%%%%%%%%%%%%%%%%%%%%%%%%%%%%%%%%%%%%%%%%%%%%%%%%%%%%%
%%																				%%
%% File name: 		abstract.tex												%%
%% Project name:	Applications in Deep Learning								%%
%% Type of work:	Advanced Seminar											%%
%% Author:			Hannes Bohnengel											%%
%% Mentor:			Debayan Roy													%%
%% Date:			11 July 2017												%%
%% University:		Technical University of Munich								%%
%% Comments:		Created in texstudio with tab width = 4						%%
%%																				%%
%%%%%%%%%%%%%%%%%%%%%%%%%%%%%%%%%%%%%%%%%%%%%%%%%%%%%%%%%%%%%%%%%%%%%%%%%%%%%%%%%%

\begin{abstract}
	Speech synthesis plays an important role in speech-based human-machine user interfaces of today's mobile devices and therefore has attracted huge research impetus in the last few years. Conventional approaches include formant-based synthesis, unit-selection synthesis and \ac*{SPSS} among others. In practice, \acs*{SPSS} is widely used due to its small model size and easily adjustable voice characeristics. However, compared to natural speech, the voice of an \acs*{SPSS}-based synthesized speech sounds muffed due to over-smoothing and therefore can be improved. On the other hand, the implementation of speech synthesis on mobile devices also presents challenges like limited memory resource, processing capacity and real-time responsiveness. Towards studying the problems of improving voice quality and embedded implementation, I will survey the priliminary works done on applying deep learning for speech synthesis. First, I will review how deep learning models can be employed to improve the voice quality and enhance prediction performance. Next, I will discuss an optimized approach for embedded implementation without compromizing much on  voice quality for a specific instance of \acs*{SPSS}. Correspondingly, I will also show how deep learning models can be applied in different stages of conventional \acs*{SPSS} to significantly reduce the memory footprint. Through my survey, I would like to bring into attention the topic of speech synthesis on mobile devices using deep learning. I believe there is still a huge scope for improvement both in terms of voice quality and efficient embedded implementation and thus this topic is a promising research direction for the future.
	
	% OLD: \ac*{SPSS} is a commonly used approach to synthesize speech and has several advantages over the alternative methods such as formant-based or unit-selection synthesis. The two most significant benefits are the small model size and the easily adjustable voice characteristics. However, the quality of the generated speech still has potential for improvement. Due to over-smoothing, the voice sounds muffled in comparison to natural speech~\cite{black:statistical}. To address this issue, the use of deep learning models has been suggested by different researchers~\cite{zen:deepstatistical, hashimoto:effect}. Throughout their experiments, a more natural voice as well as an improved prediction performance could be achieved. When implementing speech synthesis on mobile devices, different issues arise like memory constraints of mobile operating systems or the claim for real-time responsiveness. To tackle these challenges, two approaches have been introduced. The first approach includes a number of optimization steps and results in a decrease of the computation time by 65~\% without a considerable loss of voice quality~\cite{toth:optimizing}. In the second approach, a deep learning model has been deployed in different parts of a \acs*{SPSS} system~\cite{boros:robust}. The main result is a tremendous reduction of the footprint size enabling the independence of Internet-based applications. In general, it can be concluded that the use of deep learning models is a reasonable measure for speech synthesis on mobile devices.
	
	% VERY OLD: In this paper, I aim to give a systematic review about the impact of deep learning-based implementation of speech synthesis systems on resource-constrained devices. First a brief introduction into \ac{SPSS} and its alternatives is given, then two approaches on how to improve the prediction performance and the speech quality of \ac{SPSS} with deep learning models are presented. In Section~\ref{sec:embeddedspeech}, two strategies on how to adapt speech synthesis to be efficiently used on mobile platforms are outlined. Therefore the focus is set on achieving real-time responsiveness and a small memory footprint in order to get rid of the reliance on Internet-based services to apply speech synthesis on mobile devices.
	%Therefore the performance and quality is compared to conventional systems.
\end{abstract}
