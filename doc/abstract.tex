%%%%%%%%%%%%%%%%%%%%%%%%%%%%%%%%%%%%%%%%%%%%%%%%%%%%%%%%%%%%%%%%%%%%%%%%%%%%%%%%%%
%%																				%%
%% File name: 		abstract.tex												%%
%% Project name:	Applications in Deep Learning								%%
%% Type of work:	Advanced Seminar											%%
%% Author:			Hannes Bohnengel											%%
%% Mentor:			Debayan Roy													%%
%% Date:			12 May 2017													%%
%% University:		Technical University of Munich								%%
%% Comments:		Created in texstudio with tab width = 4						%%
%%																				%%
%%%%%%%%%%%%%%%%%%%%%%%%%%%%%%%%%%%%%%%%%%%%%%%%%%%%%%%%%%%%%%%%%%%%%%%%%%%%%%%%%%

\begin{abstract}
Multifrequency media access control has been well understood in general wireless ad hoc networks, while in wireless sensor networks, researchers still focus on single frequency solutions. In wireless sensor networks, each device is typically equipped with a single radio transceiver and applications adopt much smaller packet sizes compared to those in general wireless ad hoc networks. Hence, the multifrequency MAC protocols proposed for general wireless ad hoc networks are not suitable for wireless sensor network applications, which we further demonstrate through our simulation experiments. In
this article, we propose MMSN, which takes advantage of multifrequency availability while, at the same time, takes into
consideration the restrictions of wireless sensor networks. Through extensive experiments, MMSN exhibits the prominent ability to utilize parallel transmissions among neighboring nodes.
\end{abstract}