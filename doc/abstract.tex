% !TeX root = ../main.tex

%%%%%%%%%%%%%%%%%%%%%%%%%%%%%%%%%%%%%%%%%%%%%%%%%%%%%%%%%%%%%%%%%%%%%%%%%%%%%%%%%%
%%																				%%
%% File name: 		abstract.tex												%%
%% Project name:	Applications in Deep Learning								%%
%% Type of work:	Advanced Seminar											%%
%% Author:			Hannes Bohnengel											%%
%% Mentor:			Debayan Roy													%%
%% Date:			06 July 2017												%%
%% University:		Technical University of Munich								%%
%% Comments:		Created in texstudio with tab width = 4						%%
%%																				%%
%%%%%%%%%%%%%%%%%%%%%%%%%%%%%%%%%%%%%%%%%%%%%%%%%%%%%%%%%%%%%%%%%%%%%%%%%%%%%%%%%%

\begin{abstract}
	%In this paper, I aim to give a systematic review about the impact of deep learning-based  implementation of speech synthesis systems on resource-constrained devices.
	\ac*{SPSS} is a commonly used approach to synthesize speech and has several advantages over the alternative methods like formant-based or unit-selection synthesis. The two most significant advantages are the small model size and the easily adjustable voice characteristics. However, the quality of the generated speech still has potential for improvement. Due to over-smoothing the voice sounds muffled in comparison to natural speech~\cite{black:statistical}. To address this issue, the use of deep learning models has been suggested by different researchers~\cite{zen:deepstatistical, hashimoto:effect}. Throughout their experiments, a more natural voice as well as an improved prediction performance could be achieved. When implementing speech synthesis on mobile devices, there arise different issues like memory constraints of mobile operating systems or the claim for real-time responsiveness. To tackle these challenges, two approaches have been introduced. The first approach includes a number of optimization steps and results in a decrease of the computation time by 65~\% without a considerable loss of voice quality~\cite{toth:optimizing}. In the second approach a deep learning model has been deployed in different parts of a \acs*{SPSS} system~\cite{boros:robust}. Here a tremendous reduction of the footprint size is the main result and that way enables the independence on Internet-based applications. In general, it can be concluded that the use of deep learning models for speech synthesis on mobile devices is a reasonable measure.
	% OLD: In this paper, I aim to give a systematic review about the impact of deep learning-based implementation of speech synthesis systems on resource-constrained devices. First a brief introduction into \ac{SPSS} and its alternatives is given, then two approaches on how to improve the prediction performance and the speech quality of \ac{SPSS} with deep learning models are presented. In Section~\ref{sec:embeddedspeech}, two strategies on how to adapt speech synthesis to be efficiently used on mobile platforms are outlined. Therefore the focus is set on achieving real-time responsiveness and a small memory footprint in order to get rid of the reliance on Internet-based services to apply speech synthesis on mobile devices.
	%Therefore the performance and quality is compared to conventional systems.
\end{abstract}
