% !TeX root = ../main.tex

%%%%%%%%%%%%%%%%%%%%%%%%%%%%%%%%%%%%%%%%%%%%%%%%%%%%%%%%%%%%%%%%%%%%%%%%%%%%%%%%%%
%%																				%%
%% File name: 		abstract.tex												%%
%% Project name:	Applications in Deep Learning								%%
%% Type of work:	Advanced Seminar											%%
%% Author:			Hannes Bohnengel											%%
%% Mentor:			Debayan Roy													%%
%% Date:			13 July 2017												%%
%% University:		Technical University of Munich								%%
%% Comments:		Created in texstudio with tab width = 4						%%
%%																				%%
%%%%%%%%%%%%%%%%%%%%%%%%%%%%%%%%%%%%%%%%%%%%%%%%%%%%%%%%%%%%%%%%%%%%%%%%%%%%%%%%%%

\begin{abstract}
	Speech synthesis plays an important role in speech-based human-machine interfaces of today's mobile devices and therefore, has attracted huge research impetus in the last few years. Conventional approaches include formant-based synthesis, unit-selection synthesis and \ac*{HMM}-based synthesis among others. \acs*{HMM}-based synthesis is the most commonly used instance of \ac*{SPSS} and is widely used due to its small model size and easily adjustable voice characteristics. However, compared to natural speech, the voice of \acs*{HMM}-based synthesized speech sounds muffled due to over-smoothing and thus, has potential for improvement~\cite{black:statistical}. Furthermore, the implementation of speech synthesis on mobile devices also presents challenges like limited memory resource, processing capacity and real-time responsiveness. Towards studying the problems of improving voice quality and embedded implementation, I will survey the preliminary works done on applying deep learning for speech synthesis. First, I will review how deep learning models can be employed to improve the voice quality and enhance prediction performance~\cite{zen:deepstatistical, hashimoto:effect}. Next, I will discuss an optimized approach for embedded implementation without compromising much on  voice quality for an \acs*{HMM}-based synthesis system~\cite{toth:optimizing}. Correspondingly, I will also show how deep learning models can be applied in different stages of conventional \acs*{SPSS} to significantly reduce the memory footprint~\cite{boros:robust}. Through my survey, I would like to bring into attention the impact of deep learning on speech synthesis with mobile devices. I believe there is still a huge scope for improvement both in terms of voice quality and efficient embedded implementation and consequently, this topic is a promising research direction for the future.
\end{abstract}
