% !TeX root = ../main.tex

%%%%%%%%%%%%%%%%%%%%%%%%%%%%%%%%%%%%%%%%%%%%%%%%%%%%%%%%%%%%%%%%%%%%%%%%%%%%%%%%%%
%%																				%%
%% File name: 		abstract.tex												%%
%% Project name:	Applications in Deep Learning								%%
%% Type of work:	Advanced Seminar											%%
%% Author:			Hannes Bohnengel											%%
%% Mentor:			Debayan Roy													%%
%% Date:			12 June 2017												%%
%% University:		Technical University of Munich								%%
%% Comments:		Created in texstudio with tab width = 4						%%
%%																				%%
%%%%%%%%%%%%%%%%%%%%%%%%%%%%%%%%%%%%%%%%%%%%%%%%%%%%%%%%%%%%%%%%%%%%%%%%%%%%%%%%%%

\begin{abstract}
	In this paper I aim to give a systematic review about the impact of deep learning-based implementation of speech synthesis systems on resource restricted devices. First a brief introduction into \ac{SPSS} and its alternatives is given, then two approaches on how to improve the prediction performance and the speech quality of \ac{SPSS} with deep learning models are presented. In Section~\ref{sec:embeddedspeech}, two strategies on how to adapt speech synthesis to be efficiently used on mobile platforms are outlined. Therefore the focus is set on achieving real-time responsiveness and a small memory footprint in order to get rid of the reliance on Internet-based services to apply speech synthesis on mobile devices.
	%Therefore the performance and quality is compared to conventional systems.
\end{abstract}