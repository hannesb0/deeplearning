% !TeX root = ../main.tex

%%%%%%%%%%%%%%%%%%%%%%%%%%%%%%%%%%%%%%%%%%%%%%%%%%%%%%%%%%%%%%%%%%%%%%%%%%%%%%%%%%
%%																				%%
%% File name: 		introduction.tex											%%
%% Project name:	Applications in Deep Learning								%%
%% Type of work:	Advanced Seminar											%%
%% Author:			Hannes Bohnengel											%%
%% Mentor:			Debayan Roy													%%
%% Date:			11 July 2017												%%
%% University:		Technical University of Munich								%%
%% Comments:		Created in texstudio with tab width = 4						%%
%%																				%%
%%%%%%%%%%%%%%%%%%%%%%%%%%%%%%%%%%%%%%%%%%%%%%%%%%%%%%%%%%%%%%%%%%%%%%%%%%%%%%%%%%

\section{Introduction}
\label{sec:introduction}

\acp{VPA} like Siri, Cortana or Google Now start having a huge impact on the way of interacting with electronic devices like smartphones or notebooks. Up to now the \acp{VPA} help with rather simple tasks such as search queries, starting phone calls or setting a clock, but according to a recent survey from the IT research firm, Gartner~\cite{gartner:assistants}, this will change in the near future. With the Facebook Messenger it is already possible to make purchases or to order an Uber car and new use cases are expected soon. The survey also states that through the vast increase of devices in the scope of the \ac{IoT} the way of interacting with machines will go towards minimal or zero touch. Instead of interacting through common touch-displays or buttons, the user simply speaks to the device like to another person. To enable this, both speech recognition and speech synthesis are essential technologies.

In this paper, I will only focus on the speech synthesis part. A widely spread technique to synthesize human speech from a given text or from linguistic descriptions is \acf{SPSS}, also referred to as \ac{SPSG}~\cite{ling:deep}. This technique is based on the usage of \acp{HMM}. Black \textit{et al.}~\cite{black:statistical} have shown that it has several advantages over its predecessor, the concatenative speech synthesis, for example the flexibility in changing voice characteristics and a smaller memory footprint. However, the quality of the generated speech still has potential for improvement. Due to over-smoothing the voice sounds muffled in comparison to natural~speech.

This is where recent achievements in deep learning come in. Deep learning is usually referred to as a class of machine learning  tech-\break niques that achieve tasks like feature extraction or pattern analysis by using many connected layers of non-linear information processing~\cite{ling:deep, li:survey}. Since 2006, advances in the training algorithms of \acp{DNN} have enabled the field of deep learning applications to emerge~\cite{boros:robust}. Most machine learning models until then had used shallow structures, like \acp{HMM}, \acp{GMM}, \acp{CRF} or \acp{SVM}. In these structures, only one layer is responsible for generating features out of the raw input signals. While achieving quite good results with rather simple problems, they reach their limit when it comes to more complex tasks like processing human language or natural images~\cite{li:survey}. In the tutorial survey~\cite{li:survey}, the author also states four different approaches to improve speech synthesis through deep learning models, whereof three are dealing with \ac{SPSS}. One of those three approaches is described in~\cite{zen:deepstatistical}, where the authors implemented a part of the speech synthesis system by using a \ac{DNN} and observed an improved performance in predicting output features. In~\cite{hashimoto:effect}, a more general approach is conducted by investigating what effects the deployment of a \ac{DNN} on different parts of the \ac{SPSS} system has. An improvement of the naturalness of the generated speech was one of the main results.

For implementing speech synthesis on resource-constrained devices like smartphones or tablets, \ac{SPSS} is considered the best solution due to the trade-off between voice quality and acceptable footprint size~\cite{toth:optimizing}. Since the computational costs of \ac{SPSS} are often high, some optimization steps are conducted in~\cite{toth:optimizing} to make the \ac{HMM}-based speech synthesis technique \ac{SPSS} more suitable for mobile devices. These steps include reducing the size of the decision trees and introducing streaming synthesis. Going one step further, in~\cite{boros:robust} an approach to adapt \ac{SPSS} for embedded devices by using a deep learning model, an auto encoder, is employed. Four tasks (syllabification, phonetic transcription, part-of-speech tagging and lexical stress prediction) are examined and tested with the use of this deep learning model. As results, the authors highlight hugely reduced model sizes and a very close performance to the state-of-the-art models. This shows, that the usage of deep learning models for speech synthesis on embedded systems is a reasonable step, not only to improve performance and voice quality, but also towards the independence of online databases for speech synthesis.

The remaining paper is structured as follows: Section~\ref{sec:speech} first states the motivation by explaning why speech synthesis is a useful technology. Then it describes the conventional approach without deep learning models for speech synthesis and gives an overview of advantages and drawbacks of the used models and techniques. This is followed by a brief explanation of one of the most commonly used technique, \ac{SPSS}, where the paper~\cite{black:statistical} has been chosen as typically cited reference. Thereafter, two possibilities how \ac{SPSS} can be improved by deploying deep learning models are characterized. For this purpose, the papers~\cite{zen:deepstatistical} and~\cite{hashimoto:effect} are reviewed. In Section~\ref{sec:embeddedspeech}, the motivation why speech synthesis is important on mobile devices is given, followed by two examples on how speech synthesis can be implemented on an embedded system, one without~\cite{toth:optimizing} and one with deep learning models~\cite{boros:robust}. Finally, Section~\ref{sec:conclusion} summarizes the essential points of this paper and gives some future directions.